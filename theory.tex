\chapter{Background Theory}
\section{Lawvere Theories}
This section gives the definition of a Lawvere Theory along with an intuition
for how the axioms a Lawvere Theory is required to follow force it to model a
universal algebra. 

\subsection{Universal Algebras}
A universal algebra is a description of a class of algebraic structures with a
set of operations obeying a set of equational laws. For example the universal
algebra describing a monoid has two operations, a nullary operation $e$ and a
binary operation $\cdot$, which obey the following equations:
\begin{gather*}
    (a\cdot b)\cdot c = a\cdot (b\cdot c) \\
    a \cdot e = a \\
    e \cdot a = a
\end{gather*}

It is important to note that no quantifiers are used in the equations, except
implicit leading universal quantifiers over variables. This is why the identity
element must be defined as a nullary operator rather than merely a member of the
carrier set. However since these ideas are equivalent we write the operator as
though it were a value.

\subsection{Example Lawvere Theory in Category of Sets}
To produce a Lawvere Theory corresponding to this universal algebra, we
initially consider a category with sets as objects and functions as morphisms,
and will relax this constraint later. For every natural number $n$ we define the
set $T_n$ of all terms constructable using the operators of the universal
algebra (in this case $e$ and $\cdot$) and $n$ distinct free variables, say
$x_0, x_1,\ldots x_{n-1}$, quotiented by the defining equations. These sets are
the objects of the Lawvere Theory, which we will call $\mathcal{T}$.

We let functions between these objects correspond to substitutions. Given a term
$t$ in $n$ variables $x_0,\ldots x_{n-1}$, we can produce a term in $m$
variables by substituting each $x_i$ in t for an element of $T_m$, so we expect
each element of $\mathcal{T}(n, m)$ to correspond to an element of $(T_m)^n$. So
we have

\[\mathcal{T}(n,m)\cong (T_m)^n\cong ((T_m)^1)^n)\cong (\mathcal{T}(1, m))^n\]

which gives us that $T_{n+m}$ must be the coproduct of $T_n$ and
$T_m$.

The category $\mathcal{T}$ has two other important properties. Firstly because
only a trivial substitution is possible on a term with no free variables it is
obvious that $T_0$ is initial. Secondly, for every function $f: \{1,\ldots n-1\}
\rightarrow \{1,\ldots m-1\}$ there is a corresponding morphism in $\mathcal{T}$
from $T_n$ to $T_m$ that for every $0\leq i\leq n-1$ substitutes $x_i$ with
$x_{f(i)}$.

\subsection{Lawvere Theories in General}
To capture the essential properties of $\mathcal{T}$ in a more general
categorical context, we first define the category F.
\begin{align*}
    stn(n) &= \{x\in \mathbb{N} | x < n\} \\
    F_{obj} &= \mathbb{N} \\
    F_{mor}(n, m) &= stn(n)\to stn(m)
\end{align*}

Let F be the category with objects $F_{obj}$, morphisms $F_{mor}$ and identities
and composition defined as normal for functions. The category $F$ is isomorphic
to the skeleton category of finite sets.

It is easy to show that $0$ is an initial object in $F$ and that given two
objects $n$ and $m$ the following is a coproduct structure,
\[
\begin{tikzcd}
    & n\arrow[d, "\iota_1^{n,m}"] \\
    m\arrow[r, "\iota_2^{n,m}"] & n+m
\end{tikzcd}
\]
where
\begin{align*}
    \iota_1^{n,m}:stn(n)\to& stn(n+m) \\
    x\mapsto&x \\
    \\
    \iota_2^{n,m}:stn(m)\to& stn(n+m) \\
    x\mapsto&x+n
\end{align*}
A Lawvere theory is a pair $(T, L)$ where $T$ is a category and $L$ is a functor
$L : F \to T$ such that
\begin{enumerate}
    \item $L$ is a bijection on objects
    \item $L(0)$ is initial in $T$
    \item $L$ preserves coproducts, meaning the diagram
    \[
    \begin{tikzcd}
        & L(n)\arrow[d, "L(\iota_1^{n,m})"] \\
        L(m)\arrow[r, "L(\iota_2^{n,m})"] & L(n+m)
    \end{tikzcd}
    \]
    is a coproduct structure.
\end{enumerate}

\subsection{Remark on Opposites}
The definition given for Lawvere Theories is in terms for the category $F$ and
requires that the functor $L$ preserves coproducts. This formalisation
intuitively maps objects of $T$ to sets of terms that can be formed with a given
set of variables. In this model a morphism $f:A\to B$ maps to a function that
takes a term in $A$ and produces a term in $B$ by substituting each variable in
the term for terms in $B$.

An (equivalent) alternative formalisation uses the category $F^{op}$ and
requires that the functor $L$ preserves products. Under this formalisation it
is more natural to think about objects of $T^{op}$ corresponding to contexts in
which terms can be formed (although of course unsorted universal algebra
contexts are completely specified by their length). In this case we can think
about a morphism $f:B\to A$ as a way of specifying terms for all the variables
in the context $A$ using only the variables available in context $B$. 

It is obvious that these two intuitions are equivalent, but some ideas are
clearer in the first intuition while others are clearer in the second. This
distinction will become more important when we introduce Csystems, as
unfortunately much of the existing literature uses the first intuition for
Lawvere Theories and the second intuition for Csystems.

\subsection{Extracting an Algebra From a Lawvere Theory}
Given a Lawvere Theory $(T,L)$ we can produce a universal algebra that it
corresponds to. For simplicity we consider $(T^{op}, L^{op})$, and for every
natural number $n$ extract a potentially infinite set of $n$-ary operators
$L^{op}(n,1)$. Projection morphisms and the product structures in $T^{op}$ can
be used to create morphisms between contexts that apply operations on some
members and leave others unaffected. This technique combined with morphism
composition produces operators that correspond to the composition of other
operators. An infinite set of defining equations for the algebra are given by
morphism equalities.

Notice that given the set based Lawvere Theory for monoids described earlier, an
infinite family of binary operations are generated, containing operators with
behaviour equivalent to $(x,y)\mapsto x\cdot y\cdot x$ for example. The
generated structure is equivalent to a monoid but has a very different
presentation. In fact Lawvere Theories cannot be put in bijection with
presentations of universal algebras, because up to isomorphism of categories
there is a unique Lawvere Theory for any universal algebra but many possible
presentations.

\subsection{Models of Lawvere Theories}
To give a model for the algebra represented by a Lawvere Theory, say $(T, L)$,
is to give a functor $M: T^{op}\to \catname{Set}$ which preserves terminal
objects and finite products. Consequently, on objects $M(L^{op}(1))$ picks out a
single set $S$ which is the carrier set of the model. Because $M$ preserves
products $M(L^{op}(n))$ must be isomorphic to $S^n$. The operators of the model
are given by the behaviour of the functor on morphisms. For instance if the
algebra contains a nullary operator $e$, as in the case of monoids, it should
correspond to an element $e'\in T^{op}(0,1)$. Since $M(L^{op}(0))$ is a terminal
set $M(e')$ picks out a single element of $S$ as expected for a nullary
operator. Similar logic can be used to show that any $n$-ary operator in the
universal algebra must be mapped by $M$ to a function $S^n\to S$.

Any such functor necessarily gives a model that respects the defining equations
of the algebra because the functor cannot send equal morphisms to distinct
functions, and all the defining equations are represented as morphism
equalities.

\subsection{Morphisms Between Lawvere Theories}
We can further define relationships between different Lawvere Theories. Given
two Lawvere theories $A = (T_1, L_1)$ and $B = (T_2, L_2)$ a morphism
$\catname{LT}(A, B)$ is a functor $G: T_1\to T_2$ such that $L_1\circ G = L_2$.

We do not explicitly require $G$ to preserve the initial object or coproducts,
but because both $L_1$ and $L_2$ are required to it must also do so.

Morphisms between Lawvere Theories correspond to specialisation of algebraic
structures. Similarly to how all models of a group are also models of a monoid,
if we have a morphism between two Lawvere Theories $G\in \catname{LT}(A, B)$ and
$M$ a model of $B$, then we have $G^{op}\circ M$ is a model of $A$.

\subsection{The Category of Lawvere Theories}
We can now define the category of Lawvere Theories $\catname{LT}$ with every
Lawvere Theory as an object, and the set of morphisms between two Lawvere
theories defined as above. Identities and composition in $\catname{LT}$ are
defined as in the category of small categories, and the category axioms are
therefore trivial.

\section{Csystems}
\subsection{C0-systems}
A C0-system is a precursor to the definition of a Csystem, and consists of a
category, $CC$ and several pieces of additional structure:
\begin{enumerate}
    \item a function $ll: CC\to \mathbb{N}$.
    \item an object $pt\in CC$.
    \item a function $ft: CC\to CC$.
    \item for each object $X\in CC$ a morphism $p_X\in CC(X, ft(X))$.
    \item for each object $X\in CC$ and morphism $f: Y\to ft(X)$ with $ll(X) >
    0$ an object $f^*X$ and a morphism $q(f, X) : f^*X\to X$.
\end{enumerate}
which satisfy a set of axioms,
\begin{enumerate}
    \item $ll(X) = 0 \longrightarrow X=pt$.
    \item $ll(X) > 0 \longrightarrow ll(ft(X)) = ll(X) - 1$.
    \item $ft(pt) = pt$.
    \item $pt$ is a final object of CC.
    \item for any $X\in CC$ and $f: Y\to ft(X)$, $ft(f^*X)=Y$ and
    \[
    \begin{tikzcd}
        f^*X\ar[r,"{q(f,X)}"]\ar[d,"p_{f^*X}"]&X\ar[d,"p_X"] \\
        Y\ar[r,"f"]&ft(X)
    \end{tikzcd}
    \]
    commutes. This diagram is referred to as the canonical square of the
    C0-system.
    \item When $f$ is the identity on $ft(X)$, $f^*X = X$ and $q(f, X) = id_X$,
    so that the canonical square becomes
    \[
    \begin{tikzcd}
        X\ar[r,"id_X"]\ar[d,"p_X"]&X\ar[d,"p_X"] \\
        ft(X)\ar[r,"id_{ft(X)}"]&ft(X)
    \end{tikzcd}
    \]
    \item For $X\in CC$ with $ll(X)>0$, if we have $g:Z\to Y$ and $f:Y\to ft(X)$
    then in the following diagram created by combining two canonical squares,
    \[
    \begin{tikzcd}
        g^*(f^*X)\ar[r,"{q(g, f^*X)}"]\ar[d,"p_{g^*(f^*X)}"]&
        f^*X\ar[r,"{q(f,X)}"]\ar[d,"p_{f^*X}"]&
        X\ar[d,"p_X"] \\
        Z\ar[r,"g"]&
        Y\ar[r,"f"]&
        ft(X)
    \end{tikzcd}
    \]
    $(g\circ f)^*X = g^*(f^*X)$ and $q(g\circ f, X) = q(g, f^*X)\circ q(f, X)$.
\end{enumerate}

(CITE CSYSTEM PAPER)

The intuition for these axioms is given by the contexts model described
previously. Each object of $CC$ corresponds to a context of the dependent type
theory being modelled, with $pt$ as the empty context, $ll(X)$ as the number of
terms in the context $X$, $ft(X)$ being the context which is identical to $X$
but with the last term dropped and $p_X$ is a (trivial) mapping which forms the
context $ft(X)$ given the context $X$.

Finally if we have a pair of contexts $Y$ and $X$, and a mapping which forms all
but the last element of $X$ from $Y$, encoded by the morphism $f: Y\to ft(X)$,
then we can always extend $Y$ by a single element to give a context in which is
is possible to form the context $X$. The extension of $Y$ is represented by
$f^*X$ and the morphism representing the mapping that forms $X$ is $q(f, X)$.

\subsection{Csystems}
C0-systems are not quite sufficient to represent dependent type theories on
their own. When extending a context there may be many possible valid solutions,
which results in multiple, non-equivalent, C0-systems for the same dependent
type theory. To prevent this we define a Csystem to be a C0-system where $f^*X$
and $q(f, X)$ make the canonical squares into pullback squares, intuitively
requiring them to give the most general possible extension.

There is an alternative possible restriction on C0-systems to give Csystems in
terms of an additional operation and algebraic constraints upon it. The
restrictions are provably equivalent (CITE) so we deal only with the pullback
restriction here.

\section{Coq/UniMath}
\subsection{Coq}
Coq and the UniMath library have been used as the underlying theory for the
machine formalisation work. Coq is a well established dependently type
programming language and proof assistant based on the calculus of inductive
constructions, with many extensions to the strength of the type theory having
been added over time. Unlike Isabelle and similar proof assistants Coq is
constructive unless non-constructive axioms are assumed. 

\subsection{\textit{UniMath}}
UniMath is a considerably more recent project aiming to formalise the
foundations of mathematics from a Univalent point of view. It is a large library
of Coq code containing proofs of various theorems in topics including Category
Theory, Combinatorics and KTheory. Although UniMath is written in Coq, it
intentionally uses a restricted version of the Coq type theory, most notably
avoiding the use of generalised inductive types or any automated theorem
proving. Conversely, the theory is strengthened considerably by assuming the
univalence axiom, from which theorems such as function extensionality and the
negation of UIP which are not provable in Coq can be proven. This project does
not make any direct use of univalence, though function extensionality is used.

\subsection{h-levels}
Because the univalence axiom is inconsistent with general UIP it is worth
briefly discussing h-levels which allow the more restricted use of UIP. A type
$T$ is defined to be constricted if it has exactly one inhabitant, a proof of
which has type
\[isconstr(T) := \Sigma x:T.\ \Pi x':T.\ x=x'\]

A type has h-level $0$ exactly if it is constricted. A type $T$ has h-level
$n+1$ if for any two inhabitants $x, x': T$ the type of equalities between
$x=x'$ has h-level $n$.

It is fairly easy to see that types of h-level $1$ can have $0$ or $1$
inhabitants, and that the number of inhabitants of a type of h-level $2$ is not
bounded. Types of h-level $1$ are referred to as propositions, and types of
h-level $2$ are referred to as sets. Since, by definition, equalities between
elements of a set have h-level $1$ there can be at most a single proof of the
equality, so we can behave as though we have UIP when working on sets.

In this project a number of lemmas regarding h-levels are used which have been
proven in \textit{UniMath}, such as: a type of h-level $n$ also has h-level
$n+1$, if types $S$ and $T$ both have h-level $n$ then $S\times T$ also has
h-level $n$ and the type of functions into a type of h-level $n$ has h-level
$n$.

\subsection{\textit{UniMath} and \textit{TypeTheory}}
The majority of the prewritten definitions and proofs used in this project comes
directly from the \textit{UniMath} library. This includes foundational
definitions like h-levels, dependent products and sums, and the natural numbers,
along with useful lemmas about these definitions and also formalisation of many
useful notions in category theory, such as categories, small categories and
functors.

However some of the output of the \textit{UniMath} project is not yet considered
sufficiently stable to be included in the main library. This content is
available in the TypeTheory library that is also published by the
\textit{UniMath} project. A complete definition of C0-systems and Csystems is
available in this library. This project uses the C0-system definition but not
the Csystem definition because we prefer the pullback definition of a Csystem,
which is not included in the TypeTheory library.