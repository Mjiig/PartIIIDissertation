\chapter{Background Theory}
This chapter describes the formal definitions of Lawvere Theories\cite{Lawvere}
and Csystems\cite{cartmellcsystems1}\cite{cartmellcsystems2}, and attempts to
explain the intuition behind how these definitions model universal algebras and
dependent type theories respectively. It also briefly describes the tools and
libraries used in the formalisation work.

\section{Lawvere Theories}
This section gives the definition of a Lawvere Theory along with an intuition
for how Lawvere Theories model universal algebras and the intuitive
interpretation of the Lawvere Theory axioms. 

\subsection{Universal Algebras}
Universal
algebra\cite{universalalgebra}\cite{univalg2}\cite{univalg3}\cite{univalg4}
provides a general description of classes of algebraic structures with
operations obeying equational laws. For example the standard universal algebraic
description of a monoid has two operations, a nullary operation $e$ and a binary
operation~$\cdot$, which obey the following equations:
\begin{gather*}
    (a\cdot b)\cdot c = a\cdot (b\cdot c) \\
    a \cdot e = a \\
    e \cdot a = a
\end{gather*}
for $a$, $b$ and $c$ variables.

It is important to note that no quantifiers may be used in the equations, except
implicit leading universal quantifiers over variables. Consequently the obvious
approach of using an existential quantifier to enforce the existence of an
identity element is not possible. Instead the identity element is defined as a
nullary operator rather than a member of the carrier set. However since these
ideas are equivalent we write the operator as though it were a value.

The requirement that all laws be equational unfortunately prevents the
definition of certain common algebraic structures, notably fields, within
universal algebra.

\subsection{Constructing Lawvere Theories in the Category of Sets}
To produce a Lawvere Theory corresponding to a universal algebra, we initially
consider a category with sets as objects and functions as morphisms. We will
relax this constraint later in section \ref{gen_LT}. For every natural number
$n$ we define $T_n$ to be the set of all terms constructible using the operators
of the universal algebra (in the case of monoids $e$ and $\cdot$) and $n$
distinct free variables, say $x_0, x_1,\ldots x_{n-1}$, quotiented by the
defining equations. These sets are the objects of a category, which we call
$\mathcal{T}$.

We let functions between these objects correspond to substitutions. Given a term
$t$ in $n$ variables $x_0,\ldots x_{n-1}$, we can produce a term in $m$
variables by substituting each $x_i$ in $t$ for an element of $T_m$, so we
expect each element of $\mathcal{T}(T_n, T_m)$ to correspond to an element of
$(T_m)^n$. So we have

\[
\mathcal{T}(T_n, T_m)\cong
(T_m)^n\cong
((T_m)^1)^n)\cong
(\mathcal{T}(T_1, T_m))^n
\]
which gives us that $T_{n+m}$ must be the coproduct of $T_n$ and
$T_m$.

The category $\mathcal{T}$ has two other important properties. Firstly because
only a trivial substitution is possible on a term with no free variables it is
obvious that $T_0$ is initial. Secondly, for every function $f: \{0,\ldots,
n-1\} \rightarrow \{0,\ldots, m-1\}$ there is a corresponding morphism in
$\mathcal{T}$ from $T_n$ to $T_m$ that for every $0\leq i\leq n-1$ substitutes
$x_i$ with $x_{f(i)}$.

\subsection{Lawvere Theories} \label{gen_LT}
To capture the essential properties of $\mathcal{T}$ in a more general
categorical context, we first define the category F. Let F be the category with
objects $F_{obj}$, morphisms $F_{mor}$ and identities and composition defined as
normal for functions where
\begin{align*}
    stn(n) &= \{x\in \mathbb{N} | x < n\} \\
    F_{obj} &= \mathbb{N} \\
    F_{mor}(n, m) &= \{f | f: stn(n)\to stn(m)\}
\end{align*}

The category $F$ is isomorphic to the skeleton category of finite sets and
functions.

It is easy to show that $0$ is an initial object in $F$ and that given two
objects $n$ and $m$ the following is a coproduct structure,
\[
\begin{tikzcd}
    & n\arrow[d, "\iota_1^{n,m}"] \\
    m\arrow[r, "\iota_2^{n,m}"] & n+m
\end{tikzcd}
\]
where
\begin{align*}
    \iota_1^{n,m}:stn(n)\to& stn(n+m) \\
    x\mapsto&x \\
    \\
    \iota_2^{n,m}:stn(m)\to& stn(n+m) \\
    x\mapsto&x+n
\end{align*}
A Lawvere theory\cite{Lawvere} is a pair $(T, L)$ where $T$ is a category and
$L$ is a functor $L : F \to T$ such that
\begin{enumerate}
    \item $L$ is a bijection on objects
    \item $L(0)$ is initial in $T$
    \item $L$ preserves coproducts, meaning that the diagram
    \[
    \begin{tikzcd}
        & L(n)\arrow[d, "L(\iota_1^{n,m})"] \\
        L(m)\arrow[r, "L(\iota_2^{n,m})"] & L(n+m)
    \end{tikzcd}
    \]
    is a coproduct.
\end{enumerate}

\subsection{Remark on Opposites}
The definition given for Lawvere Theories is in terms of the category $F$ and
requires that the functor $L$ preserves coproducts. In this formalisation
objects of $T$ intuitively correspond to sets of terms that can be formed in a
particular context. Under this intuition a morphism $f:A\to B$ corresponds to a
function that takes a term in $A$ and produces a term in $B$ by substituting
each variable in the given term for a term in $B$.

An alternative (equivalent) formalisation uses the category $F^{op}$ and
requires that the functor $L$ preserves products. In this formalisation it
is more natural to think about objects of $T^{op}$ corresponding to contexts in
which terms can be formed (although of course unsorted universal-algebra
contexts are completely specified by their length). In this case we can think
about a morphism $f:B\to A$ as a way of specifying terms for all the variables
in the context $A$ using only the variables available in context $B$ (that is,
the morphism corresponds to a way of constructing the context $A$ from the
context $B$). 

It is obvious that these two intuitions are equivalent, but some ideas are
clearer in the first intuition while others are clearer in the second one. This
distinction will become more important when we introduce Csystems, as
unfortunately much of the existing literature uses the first intuition for
Lawvere Theories and the second intuition for Csystems.

\subsection{Extracting an Algebra From a Lawvere Theory}
Given a Lawvere Theory $(T,L)$ we can produce a universal algebra that it
corresponds to. For simplicity we consider $(T^{op}, L^{op})$, and for every
natural number $n$ extract a set of $n$-ary operators $T^{op}(n,1)$. Projection
morphisms and the product structures in $T^{op}$ can be used to create morphisms
between contexts that apply operations on some members and leave others
unaffected. For instance suppose we have a morphism $f : L^{op}(2)\to L^{op}(1)$
in $T^{op}$ which corresponds to an operator denoted by ``$\cdot$". Writing
$\{x_0,\ldots, x_{n-1}\}$ as an intuitive representation of $L^{op}(n)$ we can
construct the following diagram in $T^{op}$
\[
\begin{tikzcd}
    &\{x_0, x_1, x_2\}
    \ar[ld, "L^{op}({\pi_{1}^{2,1}})"']
    \ar[rdd, "L^{op}({\pi_{2}^{2, 1}})"]
    \ar[dddd, dashed]
    & \\
    \{x_0, x_1\}\ar[dd, "f"]&& \\
    &&\{x_2\} \\
    \{(x_0\cdot x_1)\}&& \\
    &\{(x_0\cdot x_1), x_2\}
    \ar[ul, "L^{op}({\pi_{1}^{1,1}})"]
    \ar[uur, "L^{op}({\pi_{2}^{1,1}})"']
    \ar[d,"f"]
    &\\
    &\{(x_0\cdot x_1)\cdot x_2\}&
\end{tikzcd}
\]
where $\pi_1^{n, m}$ and $\pi_2^{n, m}$ are the opposites of $\iota_1^{n,m}$ and
$\iota_2^{n,m}$ respectively, so by the preservation of products from $F^{op}$
to $T^{op}$ we have a unique morphism from $\{x_0, x_1, x_2\}$ to $\{(x_0\cdot
x_1), x_2\}$ that makes the diagram commute.

This technique combined with morphism composition produces operators that
correspond to the composition of other operators, as illustrated above. An
infinite set of defining equations for the algebra are given by morphism
equalities.

Notice that given the set based Lawvere Theory for monoids described earlier, an
infinite family of binary operations are generated, containing operators with
behaviour equivalent to $(x,y)\mapsto (x\cdot y)\cdot x$ for example. The
generated structure is equivalent to a monoid but has a very different
presentation. In fact Lawvere Theories cannot be put in bijection with
presentations of universal algebras, because up to isomorphism of categories
there is a unique Lawvere Theory for any universal algebra but many possible
presentations.

\subsection{Models of Lawvere Theories}
To give a model\cite{VoevodskyFiore} for the algebra represented by a Lawvere
Theory, say $(T, L)$, is to give a functor $M: T^{op}\to \catname{Set}$ which
preserves terminal objects and finite products. Consequently, on objects
$M(L^{op}(1))$ picks out a set $S$ which is the carrier set of the model.
Because $M$ preserves products $M(L^{op}(n))$ must be isomorphic to $S^n$. The
operators of the model are given by the behaviour of the functor on morphisms.
For instance if the algebra contains a nullary operator $e$, as in the case of
monoids, it should correspond to an element $e'\in T^{op}(0,1)$. Since
$M(L^{op}(0))$ is a terminal set $M(e')$ picks out an element of $S$ as expected
for a nullary operator. Similar logic can be used to show that any $n$-ary
operator in the universal algebra is mapped by $M$ to a function $S^n\to S$.

Any such functor necessarily gives a model that respects the defining equations
of the algebra because the functor sends equal morphisms to equal
functions, and all the defining equations are represented as morphism
equalities.

\subsection{The Category Of Lawvere Theories}
We can further define relationships between different Lawvere Theories. Given
two Lawvere Theories $A = (T_1, L_1)$ and $B = (T_2, L_2)$, a
morphism\cite{VoevodskyFiore} $\catname{LT}(A, B)$ is a functor $G: T_1\to T_2$
such that $L_1\circ G = L_2$.

We do not explicitly require $G$ to preserve the initial object or coproducts,
but because both $L_1$ and $L_2$ are required to it must also do so.

Morphisms between Lawvere Theories correspond to translations of algebraic
structures. Similarly to how given a model of a group it is possible to
construct a model of a monoid over the same carrier set, if we have a morphism
between two Lawvere Theories $G\in \catname{LT}(A, B)$ and $M$ a model of $B$,
then we have $G^{op}\circ M$ is a model of $A$.

We can now define the category of Lawvere Theories $\catname{LT}$ with every
Lawvere Theory as an object, and the set of morphisms between two Lawvere
theories defined as above. Identities and composition in $\catname{LT}$ are
defined as in the category of small categories, and the category axioms are
therefore trivial.

\section{Csystems}
This section gives the definition of C0-systems, which are then used to give the
definition of Csystems\cite{cartmellcsystems1}\cite{cartmellcsystems2}. The
intuitive correspondence between the C0-system axioms and dependent type theory
properties is briefly discussed.

\subsection{C0-systems}
A C0-system\cite{voevodskycsystems} is a precursor to the definition of a
Csystem, and consists of a category, $CC$ and several pieces of additional
structure:
\begin{enumerate}
    \item a function $ll: \text{Ob}(CC)\to \mathbb{N}$.
    \item an object $pt\in CC$.
    \item a function $ft: \text{Ob}(CC)\to \text{Ob}(CC)$.
    \item for each object $X\in CC$ a morphism $p_X\in CC(X, ft(X))$.
    \item for each object $X\in CC$ and morphism $f: Y\to ft(X)$ with $ll(X) >
    0$ an object $f^*X$ and a morphism $q(f, X) : f^*X\to X$.
\end{enumerate}
which satisfy a set of axioms,
\begin{enumerate}
    \item $ll(X) = 0 \implies X=pt$.
    \item $ll(X) > 0 \implies ll(ft(X)) = ll(X) - 1$.
    \item $ft(pt) = pt$.
    \item $pt$ is a terminal object of CC.
    \item for any $X\in CC$ and $f: Y\to ft(X)$, $ft(f^*X)=Y$ and
    \[
    \begin{tikzcd}
        f^*X\ar[r,"{q(f,X)}"]\ar[d,"p_{f^*X}"]&X\ar[d,"p_X"] \\
        Y\ar[r,"f"]&ft(X)
    \end{tikzcd}
    \]
    commutes. This diagram is referred to as the canonical square of the
    C0-system.
    \item When $f$ is the identity on $ft(X)$, $f^*X = X$ and $q(f, X) = id_X$,
    so that the canonical square becomes
    \[
    \begin{tikzcd}
        X\ar[r,"id_X"]\ar[d,"p_X"]&X\ar[d,"p_X"] \\
        ft(X)\ar[r,"id_{ft(X)}"]&ft(X)
    \end{tikzcd}
    \]
    \item For $X\in CC$ with $ll(X)>0$, if we have $g:Z\to Y$ and $f:Y\to ft(X)$
    then in the following diagram created by combining two canonical squares,
    \[
    \begin{tikzcd}
        g^*(f^*X)\ar[r,"{q(g, f^*X)}"]\ar[d,"p_{g^*(f^*X)}"]&
        f^*X\ar[r,"{q(f,X)}"]\ar[d,"p_{f^*X}"]&
        X\ar[d,"p_X"] \\
        Z\ar[r,"g"]&
        Y\ar[r,"f"]&
        ft(X)
    \end{tikzcd}
    \]
    we have $(g\circ f)^*X = g^*(f^*X)$ and $q(g\circ f, X) = q(g, f^*X)\circ
    q(f, X)$.
\end{enumerate}

The intuition for these axioms is most easily seen in terms of the contexts
intuition mentioned previously. Each object of $CC$ corresponds to a context of
the dependent type theory being modelled, with $pt$ as the empty context,
$ll(X)$ as the number of variables in the context $X$, $ft(X)$ being the context
which is identical to $X$ but with the last variable dropped and $p_X$ is the
obvious projection which forms the context $ft(X)$ given the context $X$.

Finally if we have a pair of contexts $Y$ and $X$, and a mapping which forms all
but the last element of $X$ from $Y$, encoded by the morphism $f: Y\to ft(X)$,
then we can always extend $Y$ by a single element to give a context in which is
is possible to form the context $X$. The extension of $Y$ is represented by
$f^*X$ and the morphism representing the mapping that forms $X$ is $q(f, X)$.

\subsection{Csystems}
In this axiomatisation, C0-systems are not quite sufficient to model dependent
type theories on their own. When extending a context, by forming $f^*X$, there
may be many possible valid solutions, which results in multiple, non-equivalent,
C0-systems for the same dependent type theory. To prevent this we define a
Csystem to be a C0-system where $f^*X$ and $q(f, X)$ make the canonical squares
into pullback squares, intuitively requiring them to be the most general
possible extension.

There is an alternative possible restriction on C0-systems to give Csystems in
terms of an additional operation and algebraic constraints upon it. The
restrictions are provably equivalent \cite{voevodskycsystems} so we deal only
with the pullback axiomatisation here.

\section{Coq/\textit{UniMath}}
\subsection{Coq}
Coq\cite{coq} and the \textit{UniMath} library\cite{UniMath} have been used as
the underlying theory for the machine formalisation work. Coq is a well
established dependently type programming language and proof assistant based on
the calculus of inductive constructions, with many extensions to the strength of
the type theory having been added over time. Unlike Isabelle\cite{isabelle} and
similar proof assistants, Coq is constructive unless non-constructive axioms are
assumed.

\subsection{\textit{UniMath}}
\textit{UniMath}\cite{UniMath} is a considerably more recent project aiming to
formalise the foundations of mathematics from a Univalent\cite{univalence} point
of view. It is a large library of Coq code containing proofs of various theorems
in topics including Category Theory, Combinatorics and KTheory. Although
\textit{UniMath} is written in Coq, it intentionally uses a restricted version
of the Coq type theory, most notably avoiding the use of general inductive types
or any automated theorem proving. Conversely, the theory is strengthened
considerably by assuming the univalence axiom\cite{univalence}, from which
theorems such as function extensionality and the negation of axiom UIP
(``Uniqueness of Identity Proofs'' which states that for any $x$ and $x'$ if we
have $y, y' : x=x'$ then $y=y'$) which are not provable in Coq follow. This
project does not make any direct use of univalence, though function
extensionality is used.

\subsection{h-levels}
Because the univalence axiom is inconsistent with the general form of axiom UIP
it is worth briefly discussing h-levels which allow us to work as though we have
a more restricted version of axiom UIP. A type $T$ is defined to be contractible
if it has exactly one inhabitant, a proof of which has type
\[iscontr(T) := \Sigma x:T.\ \Pi x':T.\ x=x'\]

A type has h-level $0$ exactly if it is contractible. A type $T$ has h-level
$n+1$ if for any two inhabitants $x, x': T$ the type of equalities between
$x=x'$ has h-level $n$.

It is fairly easy to see that types of h-level $1$ can have $0$ or $1$
inhabitants, and that the number of inhabitants of a type of h-level $2$ need
not be bounded. Types of h-level $1$ are referred to as propositions, and types
of h-level $2$ are referred to as sets. Since, by definition, equalities between
elements of a set have h-level $1$ there can be at most a single proof of the
equality, so we can work as though we have axiom UIP when dealing with on sets.

In this project a number of lemmas regarding h-levels are used which have been
proven in \textit{UniMath}, such as: a type of h-level $n$ also has h-level
$n+1$, if types $S$ and $T$ both have h-level $n$ then $S\times T$ also has
h-level $n$ and the type of functions into a type of h-level $n$ has h-level
$n$.

\subsection{\textit{UniMath} and \textit{TypeTheory}}
The majority of the prewritten definitions and proofs used in this project come
directly from the \textit{UniMath} library\cite{UniMath}. This includes
foundational definitions like h-levels, dependent products and sums, and the
natural numbers, along with useful lemmas about these definitions and also
formalisation of many useful notions in category theory, such as categories,
small categories and functors.

However some of the output of the \textit{UniMath} project is not yet considered
sufficiently stable to be included in the main library. This content is
available in the \textit{TypeTheory} library\cite{TypeTheory} that is also
published by the \textit{UniMath} project. A complete definition of C0-systems
and Csystems is available in this library. This project uses the C0-system
definition but not the Csystem definition because we prefer the pullback
definition of a Csystem, which is not used in the \textit{TypeTheory} library.