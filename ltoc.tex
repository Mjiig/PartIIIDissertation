\chapter{Mapping Lawvere Theories to Csystems}
This chapter describes the formalised definition of the functor from the
category of Lawvere Theories to the category of Csystems on objects. In the
interests of clarity not all the details of the proof are presented, but only
the crucial components\footnote{The full code can be found at
\url{https://github.com/Mjiig/LawvereTheoryCsystems/blob/master/LtoC.v}}.

To slightly simplify the proof Coq's context features (CITE) have been used. The
following lines of code at the beginning of the file
\begin{lstlisting}
Section LtoC.
Variable (lt : LT).
\end{lstlisting}
and at the end of the file
\begin{lstlisting}
End LtoC.
\end{lstlisting}
allow the name \lstinline|lt| to be used at any point in between, and cause all terms
defined to be abstracted over this term. Effectively within the section
\lstinline|LtoC| we define terms with respect to a constant Lawvere Theory but from
outside the section all terms are functions taking a Lawvere Theory.

\section{Underlying Category}
From a given Lawvere Theory \lstinline|lt| we extract terms corresponding to its
components to simplify subsequent terms.
\begin{lstlisting}
Definition L := LT_L lt.
Definition T := LT_T lt.
\end{lstlisting}

From these we define the underlying category we will define our Csystem over,
exactly the opposite category of the underlying category of the Lawvere Theory.
\begin{lstlisting}
Definition T_op_setcategory : setcategory.
  use tpair.
  + exact (opp_precat T).
  + split.
    - apply LT_ob_set.
    - apply homset_opp.
      apply homset_property.
Defined.
\end{lstlisting}
The definition of a C0-system actually requires that the underlying category be
a small category (called a setcategory by \textit{UniMath}), which is just a
precategory combined with proofs that both the objects and the morphisms between
any two objects form sets. We use the \textit{UniMath} defined function
\lstinline|precat_opp| to get the opposite of \lstinline|T|. Since the objects of the
opposite of \lstinline|T| are just the objects of \lstinline|T| itself, we can use the
proof that the objects of the underlying category of a Lawvere Theory form a set
(\lstinline|LT_ob_set|, which we constructed in the last chapter) to prove that the
objects of \lstinline|opp_precat T| form a set. 

To prove that the morphisms between any two objects of \lstinline|opp_precat T|
form a set we prove a lemma stating that if a category has sets of morphisms, so
does its opposite.
\begin{lstlisting}
Definition homset_opp (C : precategory) ( p : has_homsets C) :
  has_homsets (opp_precat C).
Proof.
    unfold has_homsets.
    intros.
    apply p.
Defined.
\end{lstlisting}
This follows trivially, since the morphisms from $A$ to $B$ in $T^{op}$ are
exactly the morphisms from $B$ to $A$ in $T$. After using this lemma we need
only show that \lstinline|T| has sets of morphisms, which we know to be true since it
is required that the underlying precategory of a Lawvere Theory be a category.

Because Coq is aware of the definition of \lstinline|opp_precat T| from this point
onwards it nearly always automatically expands the definition in place,
replacing objects and morphisms in \lstinline|opp_precat T| with objects and
morphisms in \lstinline|T|. Consequently all the commutative diagrams we need to
construct to build a C0-system will be inverted from their normal orientation in
the proofs in this chapter.

\section{Tower Structure}
The tower structure of a C0-system refers to the length function and the father
function, which together define a sequence of objects descending from any object
in the underlying category to an object of length zero. The length function is
defined as the inverse of the $L$ functor on objects from the Lawvere Theory
\begin{lstlisting}
Definition ll (X:T) := LT_L_inv lt X.
\end{lstlisting}

Although we do not make it explicit here, this is obviously a bijection, so the
Csystem we build will be l-bijective.

If \lstinline|ll| is a bijection then combined with the C0-system axioms it fully
specifies the behaviour of \lstinline|ft|, which we formalise as
\begin{lstlisting}
Definition ft (X:T) := L ((ll X) - 1).
\end{lstlisting}
which takes the length of \lstinline|X|, and uses \lstinline|L| to produce the object with
length one less. It is worth noting that the subtraction used here is saturating
subtraction on the natural numbers, which is total and has \lstinline|0 - 1 = 0|.

The tower structure is sufficiently powerful to capture two of the C0-system
axioms, specifically that the length of the father of an object is one less than
the length of the object and that the father of an object of length zero is the
same object. For our definitions of \lstinline|ft| and \lstinline|ll| these hold
trivially.
\begin{lstlisting}
Lemma ll_of_ft (X:T) : ll(ft X) = ll(X) - 1.
Proof.
  unfold ft, ll.
  rewrite LT_L_cancel.
  apply idpath.
Defined.
\end{lstlisting}

By unfolding the definition of \lstinline|ft| we get a goal of the form
\lstinline|ll(L (ll X - 1)) = ll(X) - 1|. Since \lstinline|ll| is just the
inverse of \lstinline|L| on objects we can cancel them on the left hand side to
get a trivial equality. 

\begin{lstlisting}
Lemma ft_on_0_is_id (X : T) (q : ll X = 0) : ft X = X.
Proof.
  unfold ft.
  rewrite q.
  cbn.
  rewrite <- q.
  rewrite LT_L_cancel2.
  apply idpath.
Defined.
\end{lstlisting}

Unfolding \lstinline|ft| we get a goal of the form \lstinline|L (ll X - 1) = X|.
Since the value of \lstinline|ll X| is known we can substitute it in to get a
goal \lstinline|L (0 - 1)| = X, which can be reduced to \lstinline|L 0| = X. If
we substitute \lstinline|ll X| for 0 we get \lstinline|L (ll X) = X| which again
allows us to cancel \lstinline|L| and \lstinline|ll| to get a trivial equality.

With these definitions we are able to construct a tower object, which is just a
tuple of a small category, a definition of \lstinline|ft|, a definition of \lstinline|ll|,
and proofs of the above two axioms.
\begin{lstlisting}
Definition T_tower_str : ltower_str T_op_setcategory.
  use tpair.
  + use tpair.
    - exact ll.
    - exact ft.
  + split.
    - apply ll_of_ft.
    - apply ft_on_0_is_id.
Defined.

Definition T_tower : ltower_precat := tpair _ T_op_setcategory T_tower_str.
\end{lstlisting}

\section{Coproducts}
In defining $p_X$ and $q$ we will make extensive use of the coproducts in
\lstinline|T|, so to simplify those constructions we first reformulate the
coproducts into a form that is somewhat easier to reason about in the context of
C0-systems. The first construction takes \lstinline|n, m, c : nat| and a proof
that \lstinline|n + m = c| and gives inclusions from \lstinline|L n| and
\lstinline|L m| to \lstinline|L c| that form a coproduct.

\begin{lstlisting}
Definition general_coprod (n m c : nat) (p : n+m = c) : 
  ∑inc: ((L n)-->(L c))×((L m)-->(L c)),
    isBinCoproduct T (L n) (L m) (L c) (pr1 inc) (pr2 inc).
Proof.
  pose (d:=LT_coprods lt n m).
  destruct p.
  use tpair.
  + split.
    - exact (#L (F_inc1 n m)).
    - exact (#L (F_inc2 n m)).
  + cbn.
    exact d.
Defined.
\end{lstlisting}

The construction is relatively trivial, being exactly the coproducts guaranteed
by the Lawvere Theory axioms under a renaming. The construction is nonetheless
useful because Coq often makes it difficult to use a specific equality within a
proof without rendering the goal ill-typed. Moving the use of the equality into
a lemma allows us to tightly control what is rewritten and avoid this.

We can specialise further still, observing that any object \lstinline|X| in
\lstinline|T| with a length greater than 0 is the coproduct of \lstinline|ft X|
and \lstinline|L 1|. We use this coproduct structure frequently so want to
construct an object representing it explicitly.

To do so we need a trivial lemma on the natural numbers which is in fact just a
specialisation of a lemma proven in \textit{UniMath}.

\begin{lstlisting}
Lemma xminusplus (x : nat) (gt0 : x > 0) : x - 1 + 1 = x.
Proof.
  rewrite (minusplusnmm x 1 gt0).
  apply idpath.
Defined.
\end{lstlisting}

We then give a pair of inclusions from \lstinline|ft X| and \lstinline|L 1| into \lstinline|X|
and prove that they have coproduct structure. 
\begin{lstlisting}
Definition x_is_coprod (X : T) (gt0 : ll X > 0) : 
  ∑inc: (ft X)-->X × (L 1)-->X, isBinCoproduct T (ft X) (L 1) X (pr1 inc) (pr2 inc).
Proof.
  pose (c:=general_coprod (ll X - 1) 1 (ll X) (xminusplus (ll X) gt0)).
  unfold ft.
  pose (q:=LT_L_cancel2 lt X).
  unfold L, ll in c.
  rewrite q in c.
  exact c.
Defined.
\end{lstlisting}

These are exactly the inclusions given by \lstinline|general_coprod| for 
\lstinline|ll X - 1| and \lstinline|1| rewritten to the desired type.

Finally we can construct an object encoding the coproduct structure. This object
hides the details of the coproduct and allows us to work with it abstractly by
presenting an interface for extracting the inclusions, coproduct object and
coproduct arrows.
\begin{lstlisting}
Definition x_coprod (X : T) (gt0 : ll X > 0) : BinCoproduct T (ft X) (L 1).
Proof.
  pose (is_coprod := x_is_coprod X gt0).
  use tpair.
    + use tpair.
      - exact X.
      - cbn.
        split.
        * exact (pr1 (pr1 is_coprod)).
        * exact (pr2 (pr1 is_coprod)).
    + cbn.
      exact (pr2 is_coprod).
Defined.
\end{lstlisting}

This object is constructed by taking \lstinline|x_is_coprod| and reconstructing
an almost identical tuple (with different shape), which us to use a large number
of existing \textit{UniMath} theorems about coproducts.

\section{Projections}
We define two different projections on $T^{op}$, one of which will become the
Csystem projection and the other complements it to give a complete product. For
the object of length 0 only the first is defined. For all other objects both are
defined, and we chose the projections to correspond to the inclusions of
\lstinline|x_coprod| in $T$. In order to differentiate between these two cases
we need to show that for any object either its length is 0 or greater than 0,
and that this disjunction is decidable. To do this we construct a disjunction
between the two possibilities for any \lstinline|X|.

\begin{lstlisting}
Definition ll_destruct (X : T) : (ll X = 0) ⨿ (ll X > 0).
Proof.
  pose (p := natchoice0 (ll X)).
  destruct p as [eq | gt].
  + left.
    exact (!eq).
  + right.
    exact gt.
Defined.
\end{lstlisting}

We use a \textit{UniMath} theorem to show that any natural number is equal to 0
or greater than zero, then destructure this disjunction to modify the relations
to the form we need.

Using \lstinline|ll_destruct| we can define \lstinline|p X| for any
\lstinline|X| which will become the Csystem projection. Therefore on the object
of length 0 it must be the identity and on all other objects it should be the
opposite of our coproduct inclusion. 
\begin{lstlisting}
Definition p (X : T) : ft X --> X.
Proof.
  destruct (ll_destruct X) as [eq0 | gt0].
  + rewrite (ft_on_0_is_id X eq0).
    exact (identity X).
  + pose (coprod := x_coprod X gt0).
    exact (BinCoproductIn1 T coprod).
Defined.
\end{lstlisting}

We use \lstinline|ll_destruct| to get a proof of either \lstinline|ll X = 0| or
\lstinline|ll X > 0|. In the first case we can rewrite our goal 
\lstinline|ft X --> X| (in $T$) to \lstinline|X --> X| and give the identity. 
In the second case we can use the proof that \lstinline|ll X > 0| to construct 
a coproduct, and give the first inclusion.

The second projection is just the corresponding inclusion, and is only defined
when \lstinline|ll X > 0| so can take a proof as argument.
\begin{lstlisting}
Definition p2 (X : T) (gt0 : ll X > 0) : L 1 --> X.
Proof.
  pose (coprod := x_coprod X gt0).
  exact (BinCoproductIn2 T coprod).
Defined.
\end{lstlisting}

We also want to be able to interchange between the names \lstinline|p, p2| and the
literal inclusions of \lstinline|x_coprod| so define a pair of equalities.
\begin{lstlisting}
Lemma p_inc (X : T) (gt0 : ll X > 0) : p X = BinCoproductIn1 T (x_coprod X gt0).

Lemma p2_inc (X : T) (gt0 : ll X > 0) : 
  p2 X gt0 = BinCoproductIn2 T (x_coprod X gt0).
\end{lstlisting}
The proofs are omitted for brevity but amount to using a proof of 
\lstinline|ll X = 0| to prove we cannot be in the \lstinline|ll X = 0| case and 
then just unfolding the definitions of \lstinline|p| and \lstinline|p2|.

\section{$f^*X$ and $q$}
Finally we define the last components of a Csystem structure, $f^*X$ and $q$.

For any object $X$ and morphism $f : ft(X)\to Y$ (in $T$) we can in fact define
$f^*X$ entirely in terms of $Y$, because there is a unique object $Z$ such that
$ft(Z) = Y$. Consequently we replace the notation $f^*X$ with $child(Y)$, which
we define as
\begin{lstlisting}
Definition child (Y : T) : T := L (ll Y + 1).
\end{lstlisting}

A number of small lemmas about \lstinline|child Y| are also proven, with obvious
proofs
\begin{lstlisting}
Lemma ll_child_gt0 (X : T) : ll (child X) > 0.
Lemma ft_ch_id (X : T) : ft (child X) = X.
Lemma ch_ft_id (X : T) (gt0 : ll X > 0) : child (ft X) = X.
\end{lstlisting}

We also define the morphism \lstinline|child_proj| which goes from \lstinline|Y|
to \lstinline|child Y|, and is exactly \lstinline|p| on \lstinline|child Y| but
with its type rewritten from \lstinline|ft (child Y) --> child Y| to
\lstinline|Y --> child Y|. It has the following definition,
\begin{lstlisting}
Definition child_proj (Y : T) : Y --> child Y := emor Y · (p (child X)).
\end{lstlisting}
where \lstinline|emor Y| is the trivial morphism from \lstinline|Y| to 
\lstinline|ft (child Y)|.

Using these definitions we define \lstinline|q|. Given arbitrary 
\lstinline|X, Y: T|, \lstinline|f : ft x --> Y| and \lstinline|gt0 : ll X > 0| 
we consider the following diagram
\[
\begin{tikzcd}
    child(Y) &
    X & 
    L(1)
    \ar[l, "p_2 X"]
    \ar[ll, bend right=20, "p_2 (child Y)"'] \\
    Y
    \ar[u, "child\_proj(Y)"] &
    ft(X)
    \ar[u, "p(X)"]
    \ar[l, "f"] &
\end{tikzcd}
\]

Since \lstinline|X| is the coproduct of \lstinline|ft X| and \lstinline|L 1|
there is a unique morphism from \lstinline|X| to \lstinline|child Y| that makes
the diagram commute, and we define \lstinline|q| to be this morphism,
\begin{lstlisting}
Definition q (X Y : T) (gt0 : ll X > 0) (f : ft(X) --> Y) : X --> child (Y).
Proof.
  pose (coprod:= x_coprod X gt0).
  exact (BinCoproductArrow T coprod (f · child_proj Y) (p2 (child Y) (ll_child_gt0 Y))).
Defined.
\end{lstlisting}

The uniqueness of \lstinline|q| is significant, as we will use it to prove that
various other morphisms are equal to \lstinline|q| when proving the C0-system
axioms.

\section{C0-system axioms}
In this section we discuss the proofs of the C0-system axioms. The types of all
the constructed Coq terms are shown, but not all the proofs have been included
due to their length. In these cases a description of the proof methodology is
given.

\subsection{An Object of Length 0 is the Unique Point}
As an extension of the tower structure, we show that the tower is pointed, that
is it has a unique bottom object. 

\begin{lstlisting}
Definition T_tower_with_p : ltower_precat_and_p := tpair _ T_tower p.

Definition T_tower_is_pointed : ispointed_type T_tower_with_p.
Proof.
  use tpair.
  + use tpair.
    - exact (L 0).
    - cbn.
      rewrite (LT_L_cancel lt).
      apply idpath.
  + cbn.
    intros.
    destruct t as [t1 t2].
    use total2_paths_f.
    - cbn.
      rewrite <- t2.
      rewrite LT_L_cancel2.
      apply idpath.
    - apply proofirrelevance.
      apply nateq.
Defined.
\end{lstlisting}

The object given is obviously \lstinline|L 0|, which it is easy to prove has
length 0 by the definition of the length function as the inverse of
\lstinline|L|. Given some other object \lstinline|t1| such that 
\lstinline|ll t1 = 0| we need to show \lstinline|t1 = L 0|. Using 
\lstinline|ll t1 = 0| we rewrite the goal to \lstinline|t1 = L (ll t1)| and
cancel \lstinline|L| and \lstinline|ll| to get a trivial equality. The final
section of the proof establishes that the proof of \lstinline|ll (L 0) = 0| is
equal to the proof of \lstinline|ll t1 = 0|, which is trivial since the
natural numbers are a set.

\subsection{The Point is a Terminal Object}
This axiom requires that \lstinline|L 0| be a terminal object in 
\lstinline|opp_precat T| which is equivalent to showing that it is an initial
object in \lstinline|T| by duality. This is one of the Lawvere Theory axioms so
the proof is trivial.

\subsection{The Canonical Square Commutes}
The \textit{TypeTheory} formalisation of C0-systems splits this axiom into three
parts. Firstly we are required to prove that for any \lstinline|Y| the child of
\lstinline|Y| has a length greater than 0. This property is not actually
included in some paper definitions of a C0-system, but without it axiom 7 cannot
be formalised. This property is a trivial consequence of the definition of
\lstinline|child|, and we proved it when we defined \lstinline|child|.

Secondly we are required to prove that for any \lstinline|Y| the father of the
child of \lstinline|Y| is \lstinline|Y|, which we have also already done when we
defined \lstinline|child|.

Finally we prove that the canonical square commutes. The proof of this fact is
fairly short, and is included inline as part of a larger proof so does not have
a name. We are required to give a term of type
\begin{lstlisting}
f · idtomor (ft (child Y)) Y (ft_ch_id Y) · p (child Y) = p X · q X Y gt0 f
\end{lstlisting}
Unfolding the definition of \lstinline|q| to a term of the form
\begin{lstlisting}
BinCoproductArrow T (x_coprod X gt0) (f · child_proj Y) (...)
\end{lstlisting}
where \lstinline|(...)| is a large irrelevant term and unfolding \lstinline|p|
to a term of the form
\begin{lstlisting}
BinCoproductIn1 T (x_coprod X gt0)
\end{lstlisting}
we get a goal which is an equality with a left hand side of the form
\begin{lstlisting}
BinCoproductIn1 T (x_coprod X gt0) · 
    BinCoproductArrow T (x_coprod X gt0) (f · child_proj Y) (...)
\end{lstlisting}
The theorem \lstinline|BinCoproductIn1Commutes| lets us rewrite this entire
expression to \lstinline|(f · child_proj Y)| which unfolds to the left hand side
of the goal equality, resulting in a trivial equality.

\subsection{$q$ Preserves Identities}
On paper this axiom requires to show that $q(id_{ft(X)}, X) = id_{ft(X)}$.
Because of the requirement to be explicit about rewrites in Coq, it is slightly
more complex in \textit{UniMath}. The term we are asked to find has the form
\begin{lstlisting}
transportf 
  (λ x : T, X --> x) 
  (ch_ft_id X gt0) 
  (q X (ft X) gt0 (identity (ft X))) 
    = identity X
\end{lstlisting}

To handle this we prove that \lstinline|q X (ft X)  gt0 (identity (ft X))| is
equal to \lstinline|identity X| transported with the same equality in the
opposite direction, i.e.
\begin{lstlisting}
transportf (λ x : T, X-->x) (!(ch_ft_id X gt0)) (identity X)
\end{lstlisting}
and that we can cancel out two reversing transports.
\begin{lstlisting}
Lemma retransport (T : UU) (a b : T) (p : a = b) (P : T -> UU) (Z : P b):
  transportf P (p) (transportf P (!p) Z) = Z.
Proof.
  destruct p.
  cbn.
  unfold idfun.
  apply idpath.
Defined.
\end{lstlisting}

To prove that $q$ is equal to the transported identity we consider the following
diagram
\[
\begin{tikzcd}
    child(ft(X)) &
    X & 
    L(1)
    \ar[l, "p_2 X"]
    \ar[ll, bend right=20, "p_2 (child(ft(X)))"'] \\
    ft(X)
    \ar[u, "child\_proj(ft(X))"] &
    ft(X)
    \ar[u, "p(X)"]
    \ar[l, "id_{ft(X)}"] &
\end{tikzcd}    
\]

Since $q$ is the unique coproduct morphism generated for this diagram, to prove
that $q$ and the transported identity are equal is equivalent to proving that
the transported identity makes this diagram commute. To do this we prove two
lemmas,
\begin{lstlisting}
Lemma id_square_commute (X : T) (eq : X = child (ft X)) : 
  p X · transportf (fun x => X --> x) eq (identity X)
    = emor (ft X) · p (child (ft X)).

Lemma id_line_commute 
  (X : T) (eq : X = child (ft X)) (gt0: ll X > 0) 
  (gt0' : ll (child (ft X)) > 0):
    p2 X gt0 · transportf (fun x => X --> x) eq (identity X)
      = p2 (child (ft X)) gt0'.
\end{lstlisting}
Both proofs are mathematically fairly simple, but in Coq are complicated by the
need to find the appropriate level of abstraction to allow replacing the
transported identity with a true identity.

\subsection{$q$ Preserves Composition}
The final C0-system axiom requires that $q$ respects composition of morphisms.
Consider the following diagram in $T$.

\[
\begin{tikzcd}[sep=huge]
    child(Z)&
    child(Y)
    \ar[l, "{q(g, child(Y))}"]&
    X
    \ar[l, "{q(f, X)}"]&
    L(1)
    \ar[l, "p_2(X)"]
    \ar[ll, bend right=15, "p_2(child(Y))"']
    \ar[lll, bend right=30, "p_2(child(Z))"']\\
    Z
    \ar[u,"child\_proj(Z)"]&
    Y
    \ar[l, "g"]
    \ar[u, "child\_proj(Y)"]&
    ft(X)
    \ar[l, "f"]
    \ar[u, "p(X)"]
\end{tikzcd}    
\]

We need to show that the morphism $q(f\cdot g, X)$ is equal to $q(f, X)\cdot
q(g, child(X))$. Again we exploit the uniqueness of $q$ to establish the
equality. The full formalisation of the proof is long due to a large amount of
required rewriting and the length of the involved terms, but the structure is
relatively simple.

First we show that $p(X)\cdot q(f, X)\cdot q(g, child(Y)) = f\cdot g\cdot
child\_proj(Z)$ which follows easily since the right hand side equals $f\cdot
child\_proj(Y)\cdot q(g, child(Y))$ by the commutativity of the left square of
the diagram, which equals the left hand side by the commutativity of the right
square of the diagram.

Secondly we show $p_2(X)\cdot q(f, X)\cdot q(g, child(Y)) = p_2 (child(Z))$.
By the definition of $q$ as a coproduct arrow we have that the triangles
formed by the top line of the diagram commute, such that the left hand side can
be rewritten to $p_2(child(Y))\cdot q(g, child(Y))$ which in turn is exactly
$p_2(child(Z))$.

Since $q(f\cdot g, X)$ is defined to be the unique morphism making these
diagrams commute it is equal to $q(f, X)\cdot q(g, child(X))$ as required. With
these axioms proven we have fully defined the C0-system.
\begin{lstlisting}
Definition LtoC_obj_C0 : lC0system.
\end{lstlisting}

\subsection{Pullbacks}
Finally we are required to show that the canonical squares of the C0-system are
pullbacks in order to show that it is a Csystem. Again this proof is long
in Coq, but relatively simple mathematically. Consider this diagram
\[
\begin{tikzcd}[sep=huge]
  e &&&\\
  &
  child(Y)&
  X
  \ar[l, "{q(f,X)}"]
  \ar[ull, bend right=10, "k"]&
  L(1)
  \ar[l, "p_2(X)"]
  \ar[ll, bend right=30, "p_2(child(Y))"]\\
  &
  Y
  \ar[uul, bend left=10, "h"]
  \ar[u, "child\_proj(Y)"']&
  ft(X)
  \ar[l, "f"]
  \ar[u, "p_X"]
\end{tikzcd}  
\]

We are required to show that there is a unique morphism from $child(Y)$ to $e$
that makes the diagram commute. Note that this is a pushout square rather than a
pullback because we are working in the opposite category to the underlying
category of the Csystem.

Because $child(Y)$ is the coproduct of $Y$ and $L(1)$ we pick the unique
morphism $z$ such that $p_2(child(Y)\cdot z = p_2(X)\cdot k$ and
$child\_proj(Y)\cdot z = h$. To show this is a valid pushout morphism it is only
required to show that it commutes with $h$ and $k$ and that it is initial with
this property. That $z$ commutes with $h$ is trivial as it is one of the
defining properties of the coproduct morphism. 

To show that $z$ commutes with $k$ we use the coproduct structure of $X$. That
$p_2(X)\cdot k = p_2(X)\cdot q(f, X)\cdot z$ follows from the definition of $q$
as we have $p_2(X)\cdot q(f, X) = p_2(child(Y))$ in general. It remains to prove
that $p_X\cdot k = p_X\cdot q(f, X)\cdot z$. By the commutativity of the
canonical squares the right hand side equals $f\cdot child\_proj(Y)\cdot z$.
Since we have already proven that $z$ commutes with $h$ we can rewrite this
$f\cdot h$, and from the givens when constructing a pushout morphism we know
that $k$ and $h$ commute with the span, so we can write this as $p_X\cdot k$, as
required.

We are also required to prove that $z$ is the unique morphism with these
properties, but this is entirely trivial on paper because any other such
morphism would also be a valid coproduct morphism out of $child(Y)$, which is
definitionally unique. In Coq the result is comparably simple, but lengthened
somewhat by various rewrites required to keep terms in useful forms.

Having proven that the canonical squares of the C0-system are pullbacks we have
constructed a CSystem, completing the definition of the functor on objects.
\begin{lstlisting}
Definition LtoC_obj : lCsystem' := tpair _ LtoC_obj_C0 LtoC_obj_has_pullbacks.
\end{lstlisting}