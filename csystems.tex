\chapter{Csystem Formalisation}
A formalisation of Csystems already exists within the \textit{TypeTheory}
library distributed by the UniMath project. Unfortunately it defines a Csystem
to be a C0-system with additional algebraic structure, whereas we need the
equivalent but simpler definition that a Csystem is a C0-systems in which the
canonical squares are pullback squares. Therefore we have separately formalised
Csystems, using the definition of a C0-system found in the \textit{TypeTheory}
library and a newly formalised constraint that the canonical squares be pullback
squares.

We first discuss some difficulties in formalising a C0-system within Coq.
Although these have largely been dealt with by the formalisation given in
\textit{TypeTheory} their effects are still visible when working with Csystems
in Coq.

\section{Difficulties Formalising C0-systems}
Consider a canonical square of some C0-system, with $X$ and $Y$ any objects such
that $ll(X) > 0$.
\[
\begin{tikzcd}
    f^*X\ar[r,"{q(f,X)}"]\ar[d,"p_{f^*X}"]&X\ar[d,"p_X"] \\
    Y\ar[r,"f"]&ft(X)
\end{tikzcd}
\]
Within a strict dependently type context this diagram is challenging to directly
represent, because it conceals a complication in the type of the morphism
$p_{f^*X}$. The diagram draws the morphism from $f^*X$ to $Y$, but the type of
$p_{f^*X}$ is defined to be a morphism from $f^*X$ to $ft(f^*X)$. Since the
axioms of C0-systems require that $Y=ft(f^*X)$ this is not an issue, and it
paper proofs it is safe to use $Y$ and $ft(f^*X)$ interchangeably without
comment. However Coq requires us to be explicit about use of known equalities to
move terms from one type to another. For instance instead of 
\verb|pX (f_star gt0 f)| (which represents $p_{f^*X}$) we could use
\begin{lstlisting}
transportf (fun Z => (f_star gt0 f) --> Z) (pX (f_star gt0 f)) eq
\end{lstlisting}
where \verb|eq| is a proof that $Y$ and $ft(f^*X)$ are equal. This is
undesirable both because it is larger than expected and because wrapping the
morphism in the \verb|transportf| function call makes it much harder to
manipulate terms involving it using standard lemmas about morphism composition.

An alternative approach is taken by the \textit{TypeTheory} library. For any
canonical square, a morphism is defined from $ft(f^*X)$ to $Y$, called 
\verb|C0emor gt0 f| where \verb|gt0| is just the proof that $ll(X)>0$. This
morphism is exactly the identity on $Y$ transported to the desired type. Using
this morphism, we can redraw the canonical square as follows,
\[
\begin{tikzcd}
    f^*X\ar[rr,"{q(f,X)}"]\ar[d,"p_{f^*X}"]&&X\ar[d,"p_X"] \\
    ft(f^*X)\ar[r,"C0emor"]&Y\ar[r,"f"]&ft(X)
\end{tikzcd}
\]
which is exactly equivalent except that all the morphisms have the expected
types so the paths can be written just as the composition of the stated
morphisms.

A similar issue arises when reasoning about axiom 6 of the C0-system axioms
which requires that in this canonical square,
\[
\begin{tikzcd}[sep=huge]
    {id_{ft(X)}^*X}
    \ar[r, "{q(id_{ft(X)}, X)}"]
    \ar[d, "p_{id_{ft(X)}^*X}"]&
    X\ar[d, "p_X"] \\
    ft(X)
    \ar[r, "id_{ft(X)}"]&
    ft(X)
\end{tikzcd}
\]
${id_{ft(X)}^*X} = X$ and $q(id_{ft(X)}, X) = id_X$. The second of these
constraints is hard to formalise, because $id_X$ is a morphism from $X$ to $X$,
but $q(id_{ft(X)}, X)$ comes from $id_{ft(X)}^*X$. Although these objects are
equal to even state this constraint we are required to transport the term, again
making it more difficult to work with.

\section{Pullback Csystems}
The formalisation of a pullback Csystem is as follows
\begin{lstlisting}
Definition has_canonical_pullbacks (CC : lC0system) := 
  ∏ (X Y : CC) (gt0: ll X > 0) (f : Y --> ft X), 
    isPullback 
      ((C0emor gt0 f) · f )
      (pX X) 
      (pX (f_star gt0 f))
      (q_of_f gt0 f)
      (C0ax5c gt0 f).

Definition lCsystem' := total2 (fun CC => has_canonical_pullbacks CC).
\end{lstlisting}

The name \verb|lCsystem'| is used to avoid collision with the definition of
\verb|lCsystem| in \textit{TypeTheory}. A \verb|lCsystem'| is a C0-system along
with a proof that its canonical squares are pullbacks.

A proof that the canonical squares are a pullback takes two objects of the
C0-system, \verb|X| and \verb|Y|, a proof that \verb|ll X > 0| and a morphism
from \verb|Y| to \verb|ft X| which is enough to specify a canonical square of
the C0-system, and returns a proof that the specified canonical square is a
pullback. \verb|isPullback| is defined in \textit{UniMath} and the components of
the pullback are constructed from terms defined in the formalisation of
C0-systems. Note that where we might expect to see \verb|f| in the pullback we
have instead used \verb|(C0emor gt0 f) · f|.

It is also worth noting that we don't need to explicitly construct or take as an
argument a proof that the canonical squares commutes, since one is necessarily
given in the C0-system structure. We don't need to worry about the choice of
proof, since Csystems are small categories so the type of morphisms between any
two objects is a set and equality proofs between them are unique.