\chapter{Introduction}
The objective of this project was to machine formalise some of the precursors
required to show that the category of Lawvere Theories and the category of
l-bijective Csystems are isomorphic, using the \textit{UniMath} Coq library. 

Lawvere Theories and Csystems are categorical models of
universal algebras and dependent type theories respectively, and the restriction
of Csystems to be l-bijective intuitively corresponds to modelling dependent
type theories which have exactly one context of each length. If only a single
context of each length exists it is obvious that a context carries no
information other than the number of variables it contains, which means the type
theory is exactly as powerful as a universal algebra. We therefore might expect
the categorical models of them to be equivalent or isomorphic.

A concrete isomorphism between these two categories has been described by
Voevodsky and Fiore. In order to machine verify this proof a formalisation in
Coq is required of both categories, the two functors making up the isomorphism
and proofs that the functors compose to give the identity in both directions. In
this project I have formalised the category of Lawvere Theories and the
definition of the functor from Lawvere Theories to Csystems on objects.

\subsection*{Notation}
In keeping with the style used by Voevodsky in (CITE)(CITE) and within
\textit{UniMath}, this dissertation represents composition in the slightly
unconventional diagrammatic order, such that for functions $(f\circ g)(x) =
g(f(x))$. In a general category, given the following diagram:
\[
\begin{tikzcd}
    A\ar[r,"f"]&B\ar[r,"g"]&C
\end{tikzcd}
\]
the composition of $f$ and $g$ is written $f\circ g$.