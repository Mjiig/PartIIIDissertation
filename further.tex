\chapter{Further Work}
There are two primary possible continuations of this work, firstly completing
the proof that the category of Lawvere Theories is isomorphic to the category of
l-bijective Csystems, and secondly formalising a similar result for multi-sorted
Lawvere Theories. This chapter briefly outlines the necessary work for both of
these extensions.

\section{Full Isomorphism Proof}
In order to prove that the category of Lawvere Theories is isomorphic to the
category of l-bijective Csystems we would first need formalisations of both
categories. This work has contributed a formalisation of the category of Lawvere
Theories, but the \textit{UniMath} project also lacks a formalisation of the
category of Csystems, both in \textit{UniMath} and in \textit{TypeTheory}. The
lack of such a formalisation is the reason this project defined a functor from
Lawvere Theories to Csystems only on objects.

With such a category defined, along with the subcategory of Csystems which are
l\nobreakdash-bijective, we could define the functor from Lawvere Theories to Csystems on
morphisms as well, and prove its functoriality.

We would then also be able to define the inverse functor from Csystems and
Csystem morphisms to Lawvere Theories and Lawvere Theory morphisms. Because we
are interested in doing so only on a subcategory of Csystems we would need to
concretely formalise the restriction on Csystems to be l\nobreakdash-bijective,
to specify exactly which Csystems the functor was defined on.

Finally, with both functors defined we would like to show that both possible
compositions of the functors give identity functors. This would formally
establish that the two categories are isomorphic.

\section{Multi-Sorted Lawvere Theories}
An extension of the result that the category of Lawvere Theories is isomorphic
to the category of l-bijective Csystems shows that the category of multi-sorted
Lawvere Theories is isomorphic to the category of list-bijective
Csystems\cite{FioreVoevodsky}. This is a strict generalisation, where the
original result can be recovered by observing that the natural numbers are
isomorphic to the type of lists of elements of a unit type.