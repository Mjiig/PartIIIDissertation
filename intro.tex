\chapter{Introduction}
Dependent type theories\cite{itt} can be seen as a generalisation of universal
algebras\cite{universalalgebra} in a natural way, as both similarly describe
sets of terms that can be formed with operators and variables, differing only in
that dependent type theories consider terms under a powerful typing relation,
whilst universal algebra terms can all be seen as having a single unstructured
type. To be more precise, universal algebras are equivalent to dependent type
theories in which a context can be described entirely by the number of variables
it contains, with no additional information about the types of those variables.

To capture this generalisation precisely as a theorem, we make use of the
categorical models of universal algebras and dependent type theories, Lawvere
Theories and Csystems respectively. The concept of the number of variables in a
context is modelled in Csystems by a length function. Dependent type theories
restricted such that contexts are described entirely by their length are
therefore modelled by Csystems restricted such that their length functions are
bijections. Specifically, the equivalence between universal algebras and this
kind of restricted dependent type theory can be captured by demonstrating an
isomorphism between the category of Lawvere Theories and the category of
l-bijective Csystems.

A concrete isomorphism has been described by Voevodsky and Fiore on
paper\cite{VoevodskyFiore}. The objective of this project was to machine
formalise some of the precursors required to show that these categories are
isomorphic, using the \textit{UniMath} Coq library\cite{coq}. 

In order to machine verify this proof a formalisation in Coq is required of
both categories, the two functors making up the isomorphism and proofs that the
functors compose to give the identity in both directions. In this project I have
formalised the category of Lawvere Theories and the definition of the functor
from Lawvere Theories to Csystems on objects.

\subsection*{Notation}
In keeping with the style used by Voevodsky
in\cite{VoevodskyFiore}\cite{voevodskycsystems} and within \textit{UniMath},
this dissertation represents composition in the slightly unconventional
diagrammatic order, such that for functions $(f\circ g)(x) = g(f(x))$. In a
general category, given the following diagram:
\[
\begin{tikzcd}
    A\ar[r,"f"]&B\ar[r,"g"]&C
\end{tikzcd}
\]
the composition of $f$ and $g$ is written $f\circ g$.