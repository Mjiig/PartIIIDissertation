\chapter{Conclusion}
This project has succeeded in formalising a proof that every Lawvere Theory can
be mapped to a Csystem, with the same underlying category and a length function
corresponding to the inverse of the Lawvere Theory's defining functor on
objects. As a part of this formalisation the notion of a Lawvere Theory as well
as the category of Lawvere Theories have been formalised.

The project illustrates that it is possible to carry out formalisation of
moderate scale mathematical proofs within the context of Coq and
\textit{UniMath} but also illustrates many of the difficulties of doing so. The
possibility for single objects to have many possible names all of which are
simultaneously relevant ends up generating terms with non-trivial quantities of
rewriting which are unintuitive to work with. Attempting to modify the
formalised proof also makes it obvious that formalisation has a high likelihood
to result in programs that are extremely brittle, and minor modifications in one
place are liable to result in breaking other terms. This is possibly magnified
by Coq's use of a tactic system in addition to explicit terms to satisfy proof
obligations. The use of tactics still generates terms, which may end up being
computationally relevant, but does so opaquely so it harder to reason about.

It may be interesting to compare the complexity of a similar formalisation
within alternative proof systems and libraries, such as the Coq \textit{HoTT}
library or the Agda \textit{HoTT-Agda} library, to investigate how much of this
difficulty is inherent to category theoretic formalisation with a background of
homotopy type theory, and how much is specific to either \textit{UniMath} or
Coq.